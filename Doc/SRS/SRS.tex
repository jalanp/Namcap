\documentclass[12pt, titlepage]{article}

\usepackage{booktabs}
\usepackage{tabularx}
\usepackage{hyperref}
\hypersetup{
    colorlinks,
    citecolor=black,
    filecolor=black,
    linkcolor=red,
    urlcolor=blue
}
\usepackage[round]{natbib}
\usepackage{enumerate}

\title{SE 3XA3: Development Plan\\Namcap}

\author{Team 2, VPB Game Studio
		\\ Prajvin Jalan (jalanp)
		\\ Vatsal Shukla (shuklv2)
		\\ Baltej Toor (toorbs)
}

\date{\today}

%% Comments

\usepackage{color}

\newif\ifcomments\commentstrue

\ifcomments
\newcommand{\authornote}[3]{\textcolor{#1}{[#3 ---#2]}}
\newcommand{\todo}[1]{\textcolor{red}{[TODO: #1]}}
\else
\newcommand{\authornote}[3]{}
\newcommand{\todo}[1]{}
\fi

\newcommand{\wss}[1]{\authornote{blue}{SS}{#1}}
\newcommand{\ds}[1]{\authornote{red}{DS}{#1}}
\newcommand{\mj}[1]{\authornote{red}{MSN}{#1}}
\newcommand{\cm}[1]{\authornote{red}{CM}{#1}}
\newcommand{\mh}[1]{\authornote{red}{MH}{#1}}

% team members should be added for each team, like the following
% all comments left by the TAs or the instructor should be addressed
% by a corresponding comment from the Team

\newcommand{\tm}[1]{\authornote{magenta}{Team}{#1}}


\begin{document}

\maketitle

\pagenumbering{roman}
\tableofcontents
\listoftables
\listoffigures

\begin{table}[bp]
\caption{\bf Revision History}
\begin{tabularx}{\textwidth}{p{3cm}p{2cm}X}
\toprule {\bf Date} & {\bf Version} & {\bf Notes}\\
\midrule
2016-10-11 & 1.0 & Addition of content to Project Drivers sections excluding Naming Conventions and Terminology\\
2016-10-11 & 1.1 & Addition of content to Project Issues sections excluding Open Issues and Ideas for Solutions\\
Date 3 & 1.2 & Notes\\
\bottomrule
\end{tabularx}
\end{table}

\newpage

\pagenumbering{arabic}

This document describes the requirements for the redevelopment project Namcap.  The template for the Software
Requirements Specification (SRS) is a subset of the Volere
template~\citep{RobertsonAndRobertson2012}.  If you make further modifications
to the template, you should explicity state what modifications were made.

\section{Project Drivers}

\subsection{The Purpose of the Project}
\paragraph{}
The purpose of this project is to redevelop the classic arcade game Pacman based on an unstructured open-source development. The Pacman conceptual redevelopment, ‘Namcap’ aims to provide the classic arcade-style experience to a broader player base. The project will be geared towards a computer implementation to provide the most access for the classical gaming interaction without having to pay large amounts for additional hardware. The project will focus on redeveloping the reference open-source project with correct programming standards, modularization, and an emphasis on formal documentation.

\subsection{The Stakeholders}

\subsubsection{The Client}
\paragraph{}
The project is being developed for an external entity that has final say on the acceptance of the redevelopment and will review the project before deployment. The external entity shows interest in the redevelopment of open source projects with emphasis on improvement of programming structure and documentation.

\subsubsection{The Customers}
\paragraph{}
The general gaming population are customers for this project. A typical gamer would have access to the internet and have the computer experience to download and run the application. 

\subsubsection{Other Stakeholders}
\paragraph{}
Another stakeholder of this project is the redevelopment team. VPB Game Studio is responsible for the development of Namcap, with emphasis on the utilization of correct programming principles and formalization of documentation throughout development.

\subsection{Mandated Constraints}

\subsubsection{Solution Constraints}
\paragraph{}
The implementation shall be based on the same mechanics as the original implementation. Mechanics such as sprit movement, collision with various in-game entities, and scoring should reflect Pacman. The project is a redevelopment and aims to stay true to the retro-arcade gaming interaction that Pacman utilized. 

\subsubsection{Implementation Environment of the Current System}
\begin{enumerate}[i]
\item All controls associated with the game mechanics shall be integrated to work with any keyboard configuration.
\item The application shall be executable on both Mac, Windows, and Linux operating systems.
\end{enumerate} 

\subsubsection{Partner or Collaborative Applications}
\paragraph{}
The implementation does not require any partner or collaborative applications to execute. The application runs independently of any external or pre-existing applications.

\subsubsection{Off-the-Shelf Software}
\paragraph{}
The implementation is self-contained and therefore requires no addition OTS software to be incorporated into the project.

\subsubsection{Anticipated Workplace Environment}
\paragraph{}
The application shall be usable on desktop and laptop computer platforms. This allows users, once the application is downloaded locally, to run the game virtually anywhere with a laptop. The implementation does not affect the workplace environment externally as no audio functionality is built-in. The execution of the implementation is only limited to the extent of laptop portability.
 
\subsubsection{Schedule Constraints}
\paragraph{}
Scheduling constraints are not applicable to this redevelopment project. However, self-imposed scheduling guidelines have been put in place for the development process. Specifically, the development team is aiming to have the initial implementation fully functioning by December 2016.

\subsubsection{Budget Constraints}
\paragraph{}
Strict budget constraints are not applicable for this project. Any possible budgeting considerations for this redevelopment project focus around the team's time management and availability. Since the redevelopment is based on an open-source implementation, any and all additional resources for the project are readily and freely available.

\subsubsection{Enterprise Constraints}
\paragraph{}
The redeveloped implementation shall follow the correct programming and development standards as well as emphasis the creation of formalized documentation.

\subsection{Naming Conventions and Terminology}

\subsection{Relevant Facts and Assumptions}

\subsubsection{Relevant Facts}
\begin{enumerate}[i]
\item The existing base implementation is approximately 1600 lines of Java code.
\item The original project is to be used as a source to conceptualize based on the requirements. No code will be used from the original project.
\end{enumerate}

\subsubsection{Business Rules}
\paragraph{}
The redevelopment team shall delegate work as to equalize the amount taken on by each member. This allows for an efficient development process and is mandated by the client of the project.

\subsubsection{Assumptions}
\paragraph{}
The redevelopment project assumes that:
\begin{enumerate}[i]
\item All required software components (IDEs, Image Editing Software) will be readily available.
\item Audio functionality shall not be implemented initially (may be implemented in the future following the appropriate decision process).
\item The implementation will be executed in a verified instance of either the Mac, Windows, or Linux operating systems.
\item The majority of users will have basic computer knowledge and experience.
\end{enumerate}

\section{Functional Requirements}

\subsection{The Scope of the Work and the Product}

\subsubsection{The Context of the Work}

\subsubsection{Work Partitioning}

\subsubsection{Individual Product Use Cases}

\subsection{Functional Requirements}

\section{Non-functional Requirements}

\subsection{Look and Feel Requirements}

\subsection{Usability and Humanity Requirements}

\subsection{Performance Requirements}

\subsection{Operational and Environmental Requirements}

\subsection{Maintainability and Support Requirements}

\subsection{Security Requirements}

\subsection{Cultural Requirements}

\subsection{Legal Requirements}

\subsection{Health and Safety Requirements}

This section is not in the original Volere template, but health and safety are
issues that should be considered for every engineering project.

\section{Project Issues}

\subsection{Open Issues}

\subsection{Off-the-Shelf Solutions}

\subsubsection{Ready-Made Products}
\paragraph{}
As this project is a redevelopment of Pacman, the emphasis is on the use of correct programming and documentation standards of an existing ready-made product (arcade Pacman).

\subsubsection{Reusable Components}
\paragraph{}
The Java Swing and Robot libraries will be used for graphical and testing components of the implementation.

\subsubsection{Products That Can Be Copied}
\paragraph{}
The original project contains the graphical assets for the original Pacman. These assets are easily modifiable for legal use in the redeveloped Namcap.

\subsection{New Problems}

\subsubsection{Effects on the Current Environment}
\paragraph{}
Assuming the user is using modern computer technology with default processing capability, the application will not affect any existing systems/environment.

\subsubsection{Effects on the Installed System}
\paragraph{}
The application is self-contained and will not coexist with any existing system.

\subsubsection{Potential User Problems}
\paragraph{}
Users of an existing system will not be affected by the introduction of this redeveloped implementation as it is meant to stand-alone from any prior implementation.

\subsubsection{Limitations in the Anticipated Implementation Environment That May Inhibit the New Product}
\paragraph{}
Inadequate processing power (minimal required) would inhibit the application from running optimally as it utilizes a graphical interface that drives the functionality.

\subsubsection{Follow-Up Problems}
\paragraph{}
Due to the implementation's minimal hardware and system requirements, future iterations of the hardware or software used to run the application will not affect the execution/functionality of the application. Since the project will be built-off of an open-source implementation and readily available online, demand and legality concerns are minimal.

\subsection{Tasks}

\subsubsection{Project Planning}
\paragraph{}
The redevelopment team will be responsible to allocate the appropriate resources to the completion of processes required for the final implementation.\\
\begin{tabular}{| l | l |}
\hline
\textbf{Task} & \textbf{Estimated Timeline} \\ \hline
Implementation Model Development & October 15 \\ \hline
Revision of Implementation Model & October 17 \\ \hline
Java Implementation Development & November 10 \\ \hline
Revision of Java Implementation & November 12 \\ \hline
Maintenance & Ongoing Semi-Annually \\
\hline
\end{tabular}

\subsubsection{Planning of the Development Process}
\paragraph{Phase 1 - Model Development:}
This phase of the redevelopment project will focus on the creation of a suitable model to blueprint the implementation process. The model development will mainly consists of conceptual design, cover all applicable functionalities of the program, and must be operational (revised) as of October 17.
\paragraph{Phase 2 - Program Development:}
This phase will focus on the actual programming of the modelled implementation in Java. This phase will require the availability of the IDE and corresponding Java packages to be utilized. Functionality and other testing will occur during this period for the operational deadline of November 12.
\paragraph{Phase 3 - Application Maintenance:}
This phase will encapsulate the ongoing maintenance and updates to the application in order to remain flexible with any issues that arise in the future. The emphasis will be on implementation optimality to ensure the game runs smoothly as hardware and software environments change over time.

\subsection{Migration to the New Product}
\paragraph{}
The game will be self-contained and therefore no applicable migration requirements and/or modifications of data need to occur for the introduction of the new implementation.

\subsection{Risks}
\paragraph{}
Since this project aims to redevelop a game, there are minimal risks involved with the execution or maintenance of a self-contained implementation.

\subsection{Costs}
\paragraph{}
Development assets will be modified/conceptualized from an open-source project with use of freely available programming and image editing software. Therefore there are no foreseen costs to the client associated with the development of this project.

\subsection{User Documentation and Training}

\subsubsection{User Documentation Requirements}
\paragraph{}
A user manual detailing the control set and in-game entity uses will be included with the final implementation. The document will list the controls required to play the game and will detail what each entity in the game does and how it affects the overall gameplay. If additional gameplay mechanics are integrated into the design in the future, the user manual will have to be updated to encapsulate the changes made such that the user has up-to-date information on how the game works.

\subsubsection{Training Requirements}
\paragraph{}
Training requirements are not applicable to this project. The user manual will include all information needed by the user to play the game.

\subsection{Waiting Room}
\paragraph{}
In-game audio, save and load game, and additional level creation functionalities will not be a part of the initial development release but may be revisited for future iterations.

\subsection{Ideas for Solutions}

\bibliographystyle{plainnat}

\bibliography{SRS}

\newpage

\section{Appendix}

This section has been added to the Volere template.  This is where you can place
additional information.

\subsection{Symbolic Parameters}

The definition of the requirements will likely call for SYMBOLIC\_CONSTANTS.
Their values are defined in this section for easy maintenance.


\end{document}
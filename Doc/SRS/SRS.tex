\documentclass[12pt, titlepage]{article}

\usepackage{booktabs}
\usepackage{tabularx}
\usepackage{hyperref}
\hypersetup{
    colorlinks,
    citecolor=black,
    filecolor=black,
    linkcolor=red,
    urlcolor=blue
}
\usepackage[round]{natbib}
\usepackage{enumerate}

\title{SE 3XA3: Development Plan\\Namcap}

\author{Team 2, VPB Game Studio
		\\ Prajvin Jalan (jalanp)
		\\ Vatsal Shukla (shuklv2)
		\\ Baltej Toor (toorbs)
}

\date{\today}

%% Comments

\usepackage{color}

\newif\ifcomments\commentstrue

\ifcomments
\newcommand{\authornote}[3]{\textcolor{#1}{[#3 ---#2]}}
\newcommand{\todo}[1]{\textcolor{red}{[TODO: #1]}}
\else
\newcommand{\authornote}[3]{}
\newcommand{\todo}[1]{}
\fi

\newcommand{\wss}[1]{\authornote{blue}{SS}{#1}}
\newcommand{\ds}[1]{\authornote{red}{DS}{#1}}
\newcommand{\mj}[1]{\authornote{red}{MSN}{#1}}
\newcommand{\cm}[1]{\authornote{red}{CM}{#1}}
\newcommand{\mh}[1]{\authornote{red}{MH}{#1}}

% team members should be added for each team, like the following
% all comments left by the TAs or the instructor should be addressed
% by a corresponding comment from the Team

\newcommand{\tm}[1]{\authornote{magenta}{Team}{#1}}


\begin{document}

\maketitle

\pagenumbering{roman}
\tableofcontents
\listoftables
\listoffigures

\begin{table}[bp]
\caption{\bf Revision History}
\begin{tabularx}{\textwidth}{p{3cm}p{2cm}X}
\toprule {\bf Date} & {\bf Version} & {\bf Notes}\\
\midrule
2016-10-11 & 1.0 & Addition of content to Project Drivers sections excluding Naming Conventions and Terminology\\
2016-10-11 & 1.1 & Addition of content to Project Issues sections excluding Open Issues and Ideas for Solutions\\
2016-10-11 & 1.2 & Addition of content to Non-functional Requirements section\\
Date 3 & 1.3 & Notes\\
\bottomrule
\end{tabularx}
\end{table}

\newpage

\pagenumbering{arabic}

This document describes the requirements for the redevelopment project Namcap.  The template for the Software
Requirements Specification (SRS) is a subset of the Volere
template~\citep{RobertsonAndRobertson2012}.  If you make further modifications
to the template, you should explicity state what modifications were made.

\section{Project Drivers}

\subsection{The Purpose of the Project}
\paragraph{}
The purpose of this project is to redevelop the classic arcade game Pacman based on an unstructured open-source development. The Pacman conceptual redevelopment, ‘Namcap’ aims to provide the classic arcade-style experience to a broader player base. The project will be geared towards a computer implementation to provide the most access for the classical gaming interaction without having to pay large amounts for additional hardware. The project will focus on redeveloping the reference open-source project with correct programming standards, modularization, and an emphasis on formal documentation.

\subsection{The Stakeholders}

\subsubsection{The Client}
\paragraph{}
The project is being developed for an external entity that has final say on the acceptance of the redevelopment and will review the project before deployment. The external entity shows interest in the redevelopment of open source projects with emphasis on improvement of programming structure and documentation.

\subsubsection{The Customers}
\paragraph{}
The general gaming population are customers for this project. A typical gamer would have access to the internet and have the computer experience to download and run the application. 

\subsubsection{Other Stakeholders}
\paragraph{}
Another stakeholder of this project is the redevelopment team. VPB Game Studio is responsible for the development of Namcap, with emphasis on the utilization of correct programming principles and formalization of documentation throughout development.

\subsection{Mandated Constraints}

\subsubsection{Solution Constraints}
\paragraph{}
The implementation shall be based on the same mechanics as the original implementation. Mechanics such as sprit movement, collision with various in-game entities, and scoring should reflect Pacman. The project is a redevelopment and aims to stay true to the retro-arcade gaming interaction that Pacman utilized. 

\subsubsection{Implementation Environment of the Current System}
\begin{enumerate}[i]
\item All controls associated with the game mechanics shall be integrated to work with any keyboard configuration.
\item The application shall be executable on both Mac, Windows, and Linux operating systems.
\end{enumerate} 

\subsubsection{Partner or Collaborative Applications}
\paragraph{}
The implementation does not require any partner or collaborative applications to execute. The application runs independently of any external or pre-existing applications.

\subsubsection{Off-the-Shelf Software}
\paragraph{}
The implementation is self-contained and therefore requires no addition OTS software to be incorporated into the project.

\subsubsection{Anticipated Workplace Environment}
\paragraph{}
The application shall be usable on desktop and laptop computer platforms. This allows users, once the application is downloaded locally, to run the game virtually anywhere with a laptop. The implementation does not affect the workplace environment externally as no audio functionality is built-in. The execution of the implementation is only limited to the extent of laptop portability.
 
\subsubsection{Schedule Constraints}
\paragraph{}
Scheduling constraints are not applicable to this redevelopment project. However, self-imposed scheduling guidelines have been put in place for the development process. Specifically, the development team is aiming to have the initial implementation fully functioning by December 2016.

\subsubsection{Budget Constraints}
\paragraph{}
Strict budget constraints are not applicable for this project. Any possible budgeting considerations for this redevelopment project focus around the team's time management and availability. Since the redevelopment is based on an open-source implementation, any and all additional resources for the project are readily and freely available.

\subsubsection{Enterprise Constraints}
\paragraph{}
The redeveloped implementation shall follow the correct programming and development standards as well as emphasis the creation of formalized documentation.

\subsection{Naming Conventions and Terminology}

\subsection{Relevant Facts and Assumptions}

\subsubsection{Relevant Facts}
\begin{enumerate}[i]
\item The existing base implementation is approximately 1600 lines of Java code.
\item The original project is to be used as a source to conceptualize based on the requirements. No code will be used from the original project.
\end{enumerate}

\subsubsection{Business Rules}
\paragraph{}
The redevelopment team shall delegate work as to equalize the amount taken on by each member. This allows for an efficient development process and is mandated by the client of the project.

\subsubsection{Assumptions}
\paragraph{}
The redevelopment project assumes that:
\begin{enumerate}[i]
\item All required software components (IDEs, Image Editing Software) will be readily available.
\item Audio functionality shall not be implemented initially (may be implemented in the future following the appropriate decision process).
\item The implementation will be executed in a verified instance of either the Mac, Windows, or Linux operating systems.
\item The majority of users will have basic computer knowledge and experience.
\end{enumerate}

\section{Functional Requirements}

\subsection{The Scope of the Work and the Product}

\subsubsection{The Context of the Work}

\subsubsection{Work Partitioning}

\subsubsection{Individual Product Use Cases}

\subsection{Functional Requirements}

\section{Non-functional Requirements}

\subsection{Look and Feel Requirements}
\begin{itemize}
	\item
	Requirement Number: NF1

	Description: The application layout shall be similar to that of the original Pacman arcade game

	Rationale: The application is intended to make the enjoyable arcade-style game of Pacman accessible to users without arcade machines (so the layout should comply with those standards)

	Fit Criterion: Testers will make a comparison between the application and the original Pacman arcade game to ensure the layout looks consistent

	Priority: High

	\item
	Requirement Number: NF2

	Description: The application color scheme shall be similar to that of the original Pacman arcade game

	Rationale: The application is intended to make the enjoyable arcade-style game of Pacman accessible to users without arcade machines (so the colors should match the original game)

	Fit Criterion: Testers will make a comparison between the application and the original Pacman arcade game to ensure the colors of the games are consistent

	Priority: High
\end{itemize}

\subsection{Usability and Humanity Requirements}
\begin{itemize}
	\item
	Requirement Number: NF3

	Description: The application shall be easy for any individual above the age of 10 to use

	Rationale: The original arcade game was accessible to all ages and was simple to understand even for elementary school students - this application should reflect that ease of use

	Fit Criterion: Part of the testers for the application will be younger children, who shoudl be able to successfully maneuver Pacman through the map (collecting points and dodging enemies)

	Priority: High

	\item
	Requirement Number: NF4

	Description: The application interface shall be in English

	Rationale: The game is intended for use by anyone who speaks English

	Fit Criterion: Testers will ensure that all parts of the application (instructions, game interface, etc.) are clearly understandable in English

	Priority: High

	\item
	Requirement Number: NF5

	Description: The game shall be easy for any player to learn within their first playthrough

	Rationale: Much of the popularity of the original arcade Pacman was in the simplicity of the game, so this redevelopment should reflect that

	Fit Criterion: Majority of testers should be able to understand how to play the game within their first playthrough (3 lives)

	Priority: High

	\item
	Requirement Number: NF6

	Description: The application shall only provide information to users that is necessary for them to enjoy the game

	Rationale: In order to keep maintain simplicity in this game as per NF7, it should be easily understandable and anything not relevant to the user's experience should be kept away from them

	Fit Criterion: Testers should not have trouble understanding the objectives of the game within their first playthrough (3 lives)

	Priority: Moderate
\end{itemize}

\subsection{Performance Requirements}
\begin{itemize}
	\item
	Requirement Number: NF7

	Description: The response time between user actions and in-game operations shall be within 1 second

	Rationale: If there is a delay in response time then users will not be able to enjoy the game to a full extent

	Fit Criterion: 90\% of tests for response time will have a delay of less than 0.5 seconds

	Priority: High

	\item
	Requirement Number: NF8

	Description: The application shall not have unexpected failures

	Rationale: The application needs to be reliable so that users can spend time enjoying the game without worrying about downtime due to internal errors

	Fit Criterion: Unexpected crashes of the application will be within 1\% of application tests

	Priority: High

	\item
	Requirement Number: NF9

	Description: The application shall be expected to operate indefinitely

	Rationale: The game is provided for free as a redevelopment of the original arcade Pacman and longetivity of the application is necessary to fulfill the product's goal - lifetime enjoyment of a classic arcade game without the need for an arcade machine

	Fit Criterion: The application is available for use as a jar file, so once the initial release is downloaded then the game shall be available to play at the user's leisure

	Priority: Moderate

\end{itemize}

\subsection{Operational and Environmental Requirements}
\begin{itemize}
	\item
	Requirement Number: NF10

	Description: Application releases shall be offered to users once per year

	Rationale: In the future there may be additional features added to the game (such as sounds, more levels, more interactions, customizable options, etc.) so releases should be kept in mind

	Fit Criterion: When new releases are provided to users, previous releases shall not be updated and there shall be no maintenance on the user's part apart from downloading the update

	Priority: Low

\end{itemize}

\subsection{Maintainability and Support Requirements}
\begin{itemize}
	\item
	Requirement Number: NF11

	Description: Bugfixes shall be provided within a few days to ensure application usability

	Rationale: Since the application code is open source, any unexpected errors in the game can be fixed by any developers so that users don't have to worry about the game crashing or behaving incorrectly (and decreasing enjoyment)

	Fit Criterion: The code for the game is updated within one business week of an error being reported

	Priority: Moderate

	\item
	Requirement Number: NF12

	Description: The application is expected to run on all operating systems that have a JVM

	Rationale: This game is being redeveloped as an application that is accessible to all users, so it should run independently of the user's platform in order to fulfill this goal

	Fit Criterion: The applicaiton will be tested on multiple platforms to ensure that it can be accessible to all users

	Priority: Moderate
\end{itemize}

\subsection{Security Requirements}
\begin{itemize}
	\item
	Requirement Number: NF13

	Description: Access to application code shall be made available through an open-source repository

	Rationale: Open-source code allows for easy maintainability of the application; any developer can efficiently fix errors in the code and improve on the design

	Fit Criterion: Changes for the code can be requested through the open-source repository, with final confirmation for changes being made by the original developers of the product (VPB Game Studio)

	Priority: Moderate
\end{itemize}

\subsection{Cultural Requirements}
\begin{itemize}
	\item
	Requirement Number: NF14

	Description: The application shall not contain symbols or text that can be seen as potentially offensive to religious or ethnic groups

	Rationale: The game is intended for use by anyone worldwide and shall not discriminate against its users, especially since this would result in a decline in user satisfaction

	Fit Criterion: During testing, application shall be verified to not contain any offensive symbols or text

	Priority: High
\end{itemize}

\subsection{Legal Requirements}
There are no legal requirements in relation to this project.

\subsection{Health and Safety Requirements}
\begin{itemize}
	\item
	Requirement Number: NF15

	Description: The application shall remind the user not to play without breaks

	Rationale: One's health comes into question whenever any software application is being used continuously for a long period of time, so it is important for users not to keep playing after a certain amount of time without taking a break in order to stay healthy

	Fit Criterion: Every 2 hours of continuous playtime will pause the application and prompt the user to take a break

	Priority: Low
\end{itemize}

\section{Project Issues}

\subsection{Open Issues}

\subsection{Off-the-Shelf Solutions}

\subsubsection{Ready-Made Products}
\paragraph{}
As this project is a redevelopment of Pacman, the emphasis is on the use of correct programming and documentation standards of an existing ready-made product (arcade Pacman).

\subsubsection{Reusable Components}
\paragraph{}
The Java Swing and Robot libraries will be used for graphical and testing components of the implementation.

\subsubsection{Products That Can Be Copied}
\paragraph{}
The original project contains the graphical assets for the original Pacman. These assets are easily modifiable for legal use in the redeveloped Namcap.

\subsection{New Problems}

\subsubsection{Effects on the Current Environment}
\paragraph{}
Assuming the user is using modern computer technology with default processing capability, the application will not affect any existing systems/environment.

\subsubsection{Effects on the Installed System}
\paragraph{}
The application is self-contained and will not coexist with any existing system.

\subsubsection{Potential User Problems}
\paragraph{}
Users of an existing system will not be affected by the introduction of this redeveloped implementation as it is meant to stand-alone from any prior implementation.

\subsubsection{Limitations in the Anticipated Implementation Environment That May Inhibit the New Product}
\paragraph{}
Inadequate processing power (minimal required) would inhibit the application from running optimally as it utilizes a graphical interface that drives the functionality.

\subsubsection{Follow-Up Problems}
\paragraph{}
Due to the implementation's minimal hardware and system requirements, future iterations of the hardware or software used to run the application will not affect the execution/functionality of the application. Since the project will be built-off of an open-source implementation and readily available online, demand and legality concerns are minimal.

\subsection{Tasks}

\subsubsection{Project Planning}
\paragraph{}
The redevelopment team will be responsible to allocate the appropriate resources to the completion of processes required for the final implementation.\\
\begin{tabular}{| l | l |}
\hline
\textbf{Task} & \textbf{Estimated Timeline} \\ \hline
Implementation Model Development & October 15 \\ \hline
Revision of Implementation Model & October 17 \\ \hline
Java Implementation Development & November 10 \\ \hline
Revision of Java Implementation & November 12 \\ \hline
Maintenance & Ongoing Semi-Annually \\
\hline
\end{tabular}

\subsubsection{Planning of the Development Process}
\paragraph{Phase 1 - Model Development:}
This phase of the redevelopment project will focus on the creation of a suitable model to blueprint the implementation process. The model development will mainly consists of conceptual design, cover all applicable functionalities of the program, and must be operational (revised) as of October 17.
\paragraph{Phase 2 - Program Development:}
This phase will focus on the actual programming of the modelled implementation in Java. This phase will require the availability of the IDE and corresponding Java packages to be utilized. Functionality and other testing will occur during this period for the operational deadline of November 12.
\paragraph{Phase 3 - Application Maintenance:}
This phase will encapsulate the ongoing maintenance and updates to the application in order to remain flexible with any issues that arise in the future. The emphasis will be on implementation optimality to ensure the game runs smoothly as hardware and software environments change over time.

\subsection{Migration to the New Product}
\paragraph{}
The game will be self-contained and therefore no applicable migration requirements and/or modifications of data need to occur for the introduction of the new implementation.

\subsection{Risks}
\paragraph{}
Since this project aims to redevelop a game, there are minimal risks involved with the execution or maintenance of a self-contained implementation.

\subsection{Costs}
\paragraph{}
Development assets will be modified/conceptualized from an open-source project with use of freely available programming and image editing software. Therefore there are no foreseen costs to the client associated with the development of this project.

\subsection{User Documentation and Training}

\subsubsection{User Documentation Requirements}
\paragraph{}
A user manual detailing the control set and in-game entity uses will be included with the final implementation. The document will list the controls required to play the game and will detail what each entity in the game does and how it affects the overall gameplay. If additional gameplay mechanics are integrated into the design in the future, the user manual will have to be updated to encapsulate the changes made such that the user has up-to-date information on how the game works.

\subsubsection{Training Requirements}
\paragraph{}
Training requirements are not applicable to this project. The user manual will include all information needed by the user to play the game.

\subsection{Waiting Room}
\paragraph{}
In-game audio, save and load game, and additional level creation functionalities will not be a part of the initial development release but may be revisited for future iterations.

\subsection{Ideas for Solutions}

\bibliographystyle{plainnat}

\bibliography{SRS}

\newpage

\section{Appendix}

This section has been added to the Volere template.  This is where you can place
additional information.

\subsection{Symbolic Parameters}

The definition of the requirements will likely call for SYMBOLIC\_CONSTANTS.
Their values are defined in this section for easy maintenance.


\end{document}
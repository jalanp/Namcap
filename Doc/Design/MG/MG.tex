\documentclass[12pt, titlepage]{article}

\usepackage{fullpage}
\usepackage[round]{natbib}
\usepackage{multirow}
\usepackage{booktabs}
\usepackage{tabularx}
\usepackage{graphicx}
\usepackage{float}
\usepackage{hyperref}
\hypersetup{
    colorlinks,
    citecolor=black,
    filecolor=red,
    linkcolor=red,
    urlcolor=blue
}

\usepackage[round]{natbib}

\newcounter{acnum}
\newcommand{\actheacnum}{AC\theacnum}
\newcommand{\acref}[1]{AC\ref{#1}}

\newcounter{ucnum}
\newcommand{\uctheucnum}{UC\theucnum}
\newcommand{\uref}[1]{UC\ref{#1}}

\newcounter{mnum}
\newcommand{\mthemnum}{M\themnum}
\newcommand{\mref}[1]{M\ref{#1}}

\title{SE 3XA3: Module Guide\\Namcap}

\author{Team 2, VPB Game Studio
		\\ Vatsal Shukla (shuklv2)
		\\ Prajvin Jalan (jalanp)
		\\ Baltej Toor (toorbs)
}

\date{\today}

%% Comments

\usepackage{color}

\newif\ifcomments\commentstrue

\ifcomments
\newcommand{\authornote}[3]{\textcolor{#1}{[#3 ---#2]}}
\newcommand{\todo}[1]{\textcolor{red}{[TODO: #1]}}
\else
\newcommand{\authornote}[3]{}
\newcommand{\todo}[1]{}
\fi

\newcommand{\wss}[1]{\authornote{blue}{SS}{#1}}
\newcommand{\ds}[1]{\authornote{red}{DS}{#1}}
\newcommand{\mj}[1]{\authornote{red}{MSN}{#1}}
\newcommand{\cm}[1]{\authornote{red}{CM}{#1}}
\newcommand{\mh}[1]{\authornote{red}{MH}{#1}}

% team members should be added for each team, like the following
% all comments left by the TAs or the instructor should be addressed
% by a corresponding comment from the Team

\newcommand{\tm}[1]{\authornote{magenta}{Team}{#1}}


\begin{document}

\maketitle

\pagenumbering{roman}
\tableofcontents
\listoftables
\listoffigures

\begin{table}[h]
\caption{\bf Revision History}
\begin{tabularx}{\textwidth}{p{3cm}p{2cm}X}
\toprule {\bf Date} & {\bf Version} & {\bf Notes}\\
\midrule
2016-11-09 & 1.0 & Addition of Introduction Section\\
2016-11-09 & 1.1 & Completion of Module Hierarchy Section\\
2016-11-10 & 1.2 & Completion of Introduction Section\\
2016-11-10 & 1.3 & Completion of Uses Hierarchy Section\\
2016-11-10 & 1.4 & Completion of Module Decomposition Section\\
2016-11-13 & 1.5 & Addition of Anticipated and Unlikely Changes\\
2016-11-13 & 1.6 & Addition of Initial Traceability Matrices\\
2016-11-13 & 1.7 & Addition of Connection Between Requirements and Design\\
2016-11-13 & 1.8 & Addition of remaining Traceability Matrix mapping, initial revision\\
2016-11-29 & 1.9 & Modifications for Ghost to Enemy refactoring\\
2016-12-07 & 2.0 & Corrections to typos and other sections based on feedback\\
2016-12-07 & 2.1 & Alteration to document structure, removed `module groups' as their own modules in decomp. based on feedback, added requirements hyper-references\\
\bottomrule
\end{tabularx}
\end{table}

\newpage

\pagenumbering{arabic}

\section{Introduction}

\subsection{Overview}
Namcap is a re-implementation of an open-source project for the classic arcade game, Pacman.

%Addressed that MIS is not responsible for explaining syntax and the correction has been made.
\subsection{Context}
This is the Module Guide (MG) document, following the Software Requirements Specification (SRS). The SRS document specifies the functional and non-functional requirements for the project, where as the MG provides a modular decomposition of the system and shows the modular structure.

  After MG is documented, the Module Interface Specification (MIS) is generated. \textcolor{red}{The MIS is responsible for explaining the semantics of exported functions for each module.}
 % \textcolor{red}{Semantics, sure but syntax no. It should be more or less a blueprint of your modules. The framework is there, implementation is not. - CM} \\

\subsection{Design Principles}

The main design principle used for this project is the Model-View-Controller architectural pattern. Therefore, the game will be separated in terms of user controls, game mechanisms, and the user interface. Using this model will allow testing for the game mechanics apart from the input and output mechanisms.

\subsection{Document Structure}

The document is organised as such:

\begin{itemize}

\item Section 2: Anticipated and Unlikely Changes to the system's implementation.

\item Section 3: Module Hierarchy, lists all modules and their hierarchy by secrets.

\item Section 4: Explains Connection Between Requirements and Design.

\item Section 5: Module Decomposition. Details for each module.

\item Section 6: Traceability Matrix

\item Section 7: Uses Hierarchy Between Modules

\end{itemize}

\section{Anticipated and Unlikely Changes} \label{SecChange}

This section lists possible changes to the system. According to the likeliness
of the change, the possible changes are classified into two
categories. Anticipated changes are listed in Section \ref{SecAchange}, and
unlikely changes are listed in Section \ref{SecUchange}.

\subsection{Anticipated Changes} \label{SecAchange}

The following are anticipated changes pertaining to the design of Namcap. The development team will approach the project with an adapted design for change methodology that will require only the one module that contains the decision elements for the change to be altered.

\begin{description}
\item[\refstepcounter{acnum} \actheacnum \label{acSprites}:] The graphical sprites for the Player and Enemy entities in the game.
\item[\refstepcounter{acnum} \actheacnum \label{acStats}:] The representation of the score and player lives on the GUI.
\item[\refstepcounter{acnum} \actheacnum \label{acHScore}:] Tracking of the high score value to be instance-independent (high score will persist from program run to program run).
\end{description}

\subsection{Unlikely Changes} \label{SecUchange}

The following are changes that are unlikely to occur as many parts of the design will potentially need to be modified based on the decisions made. The referred-to decisions primarily pertain to system architecture and core functionalities.

\begin{description}
\item[\refstepcounter{ucnum} \uctheucnum \label{ucIO}:] Input/Output devices (Input: File (high score) and Keyboard (player control), Output: FIle (high score), Screen (displays GUI)).
\item[\refstepcounter{ucnum} \uctheucnum \label{ucInput}:] There will always be a source of input data external to the software.
\end{description}

%Removed groups of modules as modules in decomposition
\section{Module Hierarchy} \label{SecMH}

This section provides an overview of the module design. Modules are summarized
in a hierarchy decomposed by secrets in Table \ref{TblMH}. The modules listed
below, which are leaves in the hierarchy tree (with the exclusion of M1 - M3 representing generalized groups of modules), are the modules that will
actually be implemented. The Enemy Module has been implemented such that it only
contains the algorithm for enemy movement in the game, so it is a generic
module that can be used for the movement of any AI character that may be
implemented in Namcap.

\begin{description}
%\item [\refstepcounter{mnum} \mthemnum \label{mHH}:] Hardware-Hiding Module
%\item [\refstepcounter{mnum} \mthemnum \label{mBH}:] Behaviour-Hiding Module
%\item [\refstepcounter{mnum} \mthemnum \label{mSD}:] Software Decision Module
\item [\refstepcounter{mnum} \mthemnum \label{mMM}:] Main Menu Module
\item [\refstepcounter{mnum} \mthemnum \label{mC}:] Character Module
\item [\refstepcounter{mnum} \mthemnum \label{mP}:] Player Module
\item [\refstepcounter{mnum} \mthemnum \label{mB}:] Board Module
\item [\refstepcounter{mnum} \mthemnum \label{mS}:] Score Module
\item [\refstepcounter{mnum} \mthemnum \label{mMV}:] Main Menu View Module
\item [\refstepcounter{mnum} \mthemnum \label{mBV}:] Board View Module
\item [\refstepcounter{mnum} \mthemnum \label{mBC}:] Button Controller Module
\item [\refstepcounter{mnum} \mthemnum \label{mKC}:] Key Controller Module
\item [\refstepcounter{mnum} \mthemnum \label{mG}:] Enemy Module
\end{description}
%\textcolor{red}{M1 - M3 should not be modules, but groups of modules - CM} \\

\begin{table}[h!]
\centering
\begin{tabular}{p{0.3\textwidth} p{0.6\textwidth}}
\toprule
\textbf{Level 1} & \textbf{Level 2}\\
\midrule

{Hardware-Hiding Module} & ~ \\
\midrule

\multirow{9}{0.3\textwidth}{Behaviour-Hiding Module}
& Main Menu\\
& Character\\
& Player\\
& Board\\
& Score\\
& Main Menu View\\ 
& Board View\\
& Button Controller\\
& Key Controller\\
\midrule

{Software Decision Module} & Enemy\\
\bottomrule

\end{tabular}
\caption{Module Hierarchy}
\label{TblMH}
\end{table}

%Corrected `usability' typo
\section{Connection Between Requirements and Design} \label{SecConnection}

The design of the system is intended to satisfy the requirements detailed in the project SRS. The connection
between requirements and modules is listed in Table \ref{TblRT}. The Traceability Matrix maps the project modules to the corresponding requirements and anticipated changes. For these purposes only requirements that specify the code and/or implementation of the redevelopment are included. Non-functional requirements NF8 - NF17 are withheld from the traceability matrix as those requirements pertain primarily to environmental, maintainability, cultural, and legal aspects of the development. More specifically, the non-functional requirements that are included in the requirements traceability matrix have been approximated in terms of the modules that represent the respective requirement. \textcolor{red}{Usability} and performance envelope the modules that implement the major mechanical and input functionalities of the application. For NF6, the matrix uses the non-leaf modules to generalize that the implementation as a whole implements the requirement.

%Reference seems to redirect to References section correctly
\section{Module Decomposition} \label{SecMD}

Modules are decomposed according to the principle of ``information hiding''
proposed by \citet{ParnasEtAl1984}. The \emph{Secrets} field in a module
decomposition is a brief statement of the design decision hidden by the
module. The \emph{Services} field specifies \emph{what} the module will do
without documenting \emph{how} to do it. For each module, a suggestion for the
implementing software is given under the \emph{Implemented By} title. If the
entry is \emph{OS}, this means that the module is provided by the operating
system or by standard programming language libraries.

\subsection{Hardware Hiding Modules}

\begin{description}
\item[Secrets:]The data structures and algorithms used to implement the virtual
  hardware.
\item[Services:]Serves as a virtual hardware used by the rest of the
  system. This module provides the interface between the hardware and the
  software. Furthermore, the system can use it to display outputs or to accept inputs.
\item[Implemented By:] OS
\end{description}

\subsection{Behaviour-Hiding Module}

\begin{description}
\item[Secrets:]The contents of the required behaviours.
\item[Services:]Includes programs that provide externally visible behaviour of
  the system as specified in the software requirements specification (SRS)
  document. This module serves as a communication layer between the
  hardware-hiding module and the software decision module. The programs in this
  module will need to change if there are changes in the SRS.
\item[Implemented By:] --
\end{description}

\subsubsection{Main Menu Module (\mref{mMM})}

\begin{description}
\item[Secrets:]The instantiation of a new game.
\item[Services:]Starts a new game when user clicks on Start Game button.
\item[Implemented By:] Namcap
\end{description}

\subsubsection{Character Module (\mref{mC})}

\begin{description}
\item[Secrets:]Attributes of a character.
\item[Services:]Acts as a super-class to the Player and Enemy modules. Provides the character's location on the map/grid.
\item[Implemented By:] Namcap
\end{description}

\subsubsection{Player Module (\mref{mP})}

\begin{description}
\item[Secrets:]The movement mechanics of the Player.
\item[Services:]Allows the user to navigate the player across the map on a valid path.
\item[Implemented By:] Namcap
\end{description}

\subsubsection{Board Module (\mref{mB})}

\begin{description}
\item[Secrets:]The barrier detection mechanism.
\item[Services:]Provides a layout of all the barriers on the map and a layout of all the dots on the map.
\item[Implemented By:] Namcap
\end{description}

\subsubsection{Score Module (\mref{mS})}

\begin{description}
\item[Secrets:]The scoring mechanism.
\item[Services:]Provides the user's score.
\item[Implemented By:] Namcap
\end{description}

\subsubsection{Main Menu View Module (\mref{mMV})}

\begin{description}
\item[Secrets:]The display of the main menu.
\item[Services:]Displays the main menu to the user.
\item[Implemented By:] Namcap
\end{description}

\subsubsection{Board View Module (\mref{mBV})}

\begin{description}
\item[Secrets:]The main game display.
\item[Services:]Displays the game's board/map to the user.
\item[Implemented By:] Namcap
\end{description}

\subsubsection{Button Controller Module (\mref{mBC})}

\begin{description}
\item[Secrets:]The actions performed by any buttons.
\item[Services:]Performs the action. 
\item[Implemented By:] Namcap
\end{description}

\subsubsection{Key Controller Module (\mref{mKC})}

\begin{description}
\item[Secrets:]The actions performed by key presses.
\item[Services:]Provides a direction, based on what key is pressed, to the player's movement module.
\item[Implemented By:] Namcap
\end{description}


\subsection{Software Decision Module}
%\textcolor{red}{I really like that you define what these are, just in the wrong place - CM} \\
\begin{description}
\item[Secrets:] The design decision based on mathematical theorems, physical
  facts, or programming considerations. The secrets of this module are
  \emph{not} described in the SRS.
\item[Services:] Includes data structure and algorithms used in the system that
  do not provide direct interaction with the user. 
  % Changes in these modules are more likely to be motivated by a desire to
  % improve performance than by externally imposed changes.
\item[Implemented By:] --
\end{description}
%Addressed `an' typo
\subsubsection{Enemy Module (\mref{mG})}

\begin{description}
\item[Secrets:]The enemy's movement mechanism.
\item[Services:]Implements a simple AI that moves the enemy on a valid path.
\item[Implemented By:] Namcap
\end{description}


\section{Traceability Matrix} \label{SecTM}

This section shows two traceability matrices: between the modules and the
requirements and between the modules and the anticipated changes.

%Used hypertarget to add traceability to SRS document for individual requirements
\begin{table}[H]
\centering
\begin{tabular}{p{0.2\textwidth} p{0.6\textwidth}}
\toprule
\textbf{Req.} & \textbf{Modules}\\
\midrule
\href{../../SRS/SRS.pdf#f1}{F1} & \mref{mMM}, \mref{mMV}, \mref{mBC}\\
\href{../../SRS/SRS.pdf#f2}{F2} & \mref{mC}, \mref{mP}, \mref{mB}, \mref{mBV}, \mref{mKC}\\
\href{../../SRS/SRS.pdf#f3}{F3} & \mref{mC}, \mref{mP}, \mref{mBV}\\
\href{../../SRS/SRS.pdf#f4}{F4} & \mref{mP}, \mref{mBV}\\
\href{../../SRS/SRS.pdf#f5}{F5} & \mref{mC}, \mref{mG}\\
\href{../../SRS/SRS.pdf#f6}{F6} & \mref{mC}, \mref{mBV}, \mref{mG}\\
\href{../../SRS/SRS.pdf#f7}{F7} & \mref{mP}, \mref{mB}, \mref{mBV}\\
\href{../../SRS/SRS.pdf#f8}{F8} & \mref{mP}, \mref{mB}, \mref{mBV}\\
\href{../../SRS/SRS.pdf#f9}{F9} & \mref{mP}, \mref{mBV}\\
\href{../../SRS/SRS.pdf#f10}{F10} & \mref{mP}, \mref{mB}, \mref{mBV}\\
\href{../../SRS/SRS.pdf#f11}{F11} & \mref{mP}, \mref{mKC}\\
\href{../../SRS/SRS.pdf#f12}{F12} & \mref{mKC}\\
\href{../../SRS/SRS.pdf#f13}{F13} & \mref{mP}, \mref{mS}\\
\href{../../SRS/SRS.pdf#f14}{F14} & \mref{mP}, \mref{mB}, \mref{mS}\\
\href{../../SRS/SRS.pdf#f15}{F15} & \mref{mP}, \mref{mBV}\\
\href{../../SRS/SRS.pdf#nf1}{NF1} & \mref{mMV}, \mref{mBV}\\
\href{../../SRS/SRS.pdf#nf2}{NF2} & \mref{mMV}, \mref{mBV}\\
\href{../../SRS/SRS.pdf#nf3}{NF3} & \mref{mC}, \mref{mP}, \mref{mB}, \mref{mBV}, \mref{mKC}\\
\href{../../SRS/SRS.pdf#nf4}{NF4} & \mref{mMV}, \mref{mBV}\\
\href{../../SRS/SRS.pdf#nf5}{NF5} & \mref{mC}, \mref{mP}, \mref{mB}, \mref{mS}, \mref{mBV}, \mref{mKC}\\
\href{../../SRS/SRS.pdf#nf6}{NF6} & \mref{mC}, \mref{mP}, \mref{mB}, \mref{mS}, \mref{mBV}, \mref{mKC}, \mref{mG}\\
\href{../../SRS/SRS.pdf#nf7}{NF7} & \mref{mC}, \mref{mP}, \mref{mB}, \mref{mBV}, \mref{mKC}\\
\bottomrule
\end{tabular}
\caption{Trace Between Requirements and Modules}
\label{TblRT}
\end{table}

\begin{table}[H]
\centering
\begin{tabular}{p{0.2\textwidth} p{0.6\textwidth}}
\toprule
\textbf{AC} & \textbf{Modules}\\
\midrule
\acref{acSprites} & \mref{mBV}\\
\acref{acStats} & \mref{mP}, \mref{mB}, \mref{mS}, \mref{mBV}\\
\acref{acHScore} & \mref{mB}, \mref{mS}, \mref{mBV}\\
\bottomrule
\end{tabular}
\caption{Trace Between Anticipated Changes and Modules}
\label{TblACT}
\end{table}

%Addressed `separately' typo
\section{Use Hierarchy Between Modules} \label{SecUse}

In this section, the uses hierarchy between modules is
provided. \citet{Parnas1978} said of two programs A and B that A {\em uses} B if
correct execution of B may be necessary for A to complete the task described in
its specification. That is, A {\em uses} B if there exist situations in which
the correct functioning of A depends upon the availability of a correct
implementation of B.  Figure \ref{FigUH} illustrates the use relation between
the modules. It can be seen that the graph is a directed acyclic graph
(DAG). Each level of the hierarchy offers a testable and usable subset of the
system, and modules in the higher level of the hierarchy are essentially simpler
because they use modules from the lower levels.\\\\
\noindent It should be noted that the design of the modules follows the model-view-controller
architectural pattern, thus the Uses Hierarchy between modules has been colour-coded to differentiate
the components. Red entities are controller modules (dealing with user input), yellow entities are view 
modules (dealing with display), and green entities are model modules (dealing with game mechanisms).
To ensure clarity, since all modules are connected to the Hardware-Hiding module - which directly interacts
with system hardware for input and output - it has been shown \textcolor{red}{separately} in the figure. \\
%\textcolor{red}{I like the design overall. If the board does in fact, hold / contain the player and enemies, it does work. An exception to this is if a player is abstracted from the level and can function on its own. Also, the model never interfaces with the view. The controller \textit{controls} how the model is displayed and thus should handle that - CM} \\
\begin{figure}[H]
\centering
\includegraphics[width=1.0\textwidth]{UsesHierarchy.png}
\caption{Use hierarchy among modules}
\label{FigUH}
\end{figure}

%\section*{References}
\newpage
\bibliographystyle {plainnat}
\bibliography {MG}

\end{document}
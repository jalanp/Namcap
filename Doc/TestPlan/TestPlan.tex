\documentclass[12pt, titlepage]{article}

\usepackage{booktabs}
\usepackage{tabularx}
\usepackage{hyperref}
\hypersetup{
    colorlinks,
    citecolor=black,
    filecolor=black,
    linkcolor=red,
    urlcolor=blue
}
%\usepackage[round]{natbib}
\usepackage{float}

\title{SE 3XA3: Test Plan\\Namcap}

\author{Team 2, VPB Game Studio
		\\ Prajvin Jalan (jalanp)
		\\ Vatsal Shukla (shuklv2)
		\\ Baltej Toor (toorbs)
}

\date{\today}

%% Comments

\usepackage{color}

\newif\ifcomments\commentstrue

\ifcomments
\newcommand{\authornote}[3]{\textcolor{#1}{[#3 ---#2]}}
\newcommand{\todo}[1]{\textcolor{red}{[TODO: #1]}}
\else
\newcommand{\authornote}[3]{}
\newcommand{\todo}[1]{}
\fi

\newcommand{\wss}[1]{\authornote{blue}{SS}{#1}}
\newcommand{\ds}[1]{\authornote{red}{DS}{#1}}
\newcommand{\mj}[1]{\authornote{red}{MSN}{#1}}
\newcommand{\cm}[1]{\authornote{red}{CM}{#1}}
\newcommand{\mh}[1]{\authornote{red}{MH}{#1}}

% team members should be added for each team, like the following
% all comments left by the TAs or the instructor should be addressed
% by a corresponding comment from the Team

\newcommand{\tm}[1]{\authornote{magenta}{Team}{#1}}


\begin{document}

\maketitle

\pagenumbering{roman}
\tableofcontents
\listoftables
\listoffigures

\begin{table}[bp]
\caption{\bf Revision History}
\begin{tabularx}{\textwidth}{p{3cm}p{2cm}X}
\toprule {\bf Date} & {\bf Version} & {\bf Notes}\\
\midrule
2016-10-30 & 1.0 & Addition of content to sections 1.1, 1.2 and 1.4\\
2016-10-30 & 1.1 & Addition of content to sections 2.1 and 2.2\\
2016-10-30 & 1.2 & Addition of content to sections 2.3 and 2.4\\
2016-10-30 & 1.3 & Addition of content to section 3.2\\
\bottomrule
\end{tabularx}
\end{table}

\newpage

\pagenumbering{arabic}

This document is the test plan for the Pacman redevelopment project, Namcap. The test plan outlines the testin methodologies and techniques to be used when testing the functionalities and characteristics of the system and its component parts.

\section{General Information}

\subsection{Purpose}
\paragraph{}
The purpose of testing is to ensure that the developed implementation functions correctly and to address any areas where the system is vulnerable. Through the formal specification of the testing methods and verification techniques, testing the implementation becomes mroe reliable.

\subsection{Scope}
\paragraph{}
As Namcap is a redevelopment of a classic arcade game, the test plan will aim to formalize the various functionality testing techniques as well as the usability tests utilized in order to ensure that the implementation meets the given requirement. This document will explicitly detail the different methods and testing tools to be utilized for this project.

\subsection{Acronyms, Abbreviations, and Symbols}
	
\begin{table}[hbp]
\caption{\textbf{Table of Abbreviations}} \label{Table}

\begin{tabularx}{\textwidth}{p{3cm}X}
\toprule
\textbf{Abbreviation} & \textbf{Definition} \\
\midrule
Abbreviation1 & Definition1\\
Abbreviation2 & Definition2\\
\bottomrule
\end{tabularx}

\end{table}

\begin{table}[!htbp]
\caption{\textbf{Table of Definitions}} \label{Table}

\begin{tabularx}{\textwidth}{p{3cm}X}
\toprule
\textbf{Term} & \textbf{Definition}\\
\midrule
Term1 & Definition1\\
Term2 & Definition2\\
\bottomrule
\end{tabularx}

\end{table}	

\subsection{Overview of Document}
\paragraph{}
This document will describe the testing methodologies to be utilized to verify Namcap as an implementation and development. The test plan will outline all testing tools, schedules, automated and manual tests, tests to address the requirements for the application, unit tests, and any additional testing performed on the PoC and existing implementation.

\section{Plan}
	
\subsection{Software Description}
\paragraph{}
Namcap is a redevelopment of the classic 2D arcade game Pacman. The gameplay involves the player sprite moving through a 2D level attempting to acquire (collide with) as many dots as possible to increase the score. Enemy sprites (ghosts in the original) will move throughout the level and upon collision with the player will cause the player to lose a life and/or end the game. The player can however consume (collide with) a power up to send the enemies back to the center of the level (when collision occurs). The implementation covers these aspects of the core gameplay (scoring, collision, movement, and enemy mechanics).

\subsection{Test Team}
\paragraph{}
The test team for Namcap is comprised of Prajvin Jalan, Vatsal Shukla, and Baltej Toor.

\subsection{Automated Testing Approach}
\paragraph{}
For the purposes of automated testing, the test team will use both JUnit and the built-in Robot class library. JUnit is a unit testing framework that will run automated tests for most logical components and any GUI components where applicable. The Robot class library will be used to automate testing that simulates user input. Based on the simulated user input, logical and GUI components will be tested to ensure that the appropriate collision and scoring responses occur.

\subsection{Testing Tools}
\paragraph{}
The automated unit testing tool JUnit and the Robot class library are the only testing tools that the team will use to verify the implementation.

\subsection{Testing Schedule}
		
Follow this link to the '\href{run:../DevelopmentPlan/NamcapGanttProject.gan}{Namcap Gantt Project}' (must have project file structure).

\section{System Test Description}
	
\subsection{Tests for Functional Requirements}

\subsubsection{Game Funcionality Testing}
		
\paragraph{}
A robot (automated) unit testing class will be implemented and used to test the mechanics of the game.

\begin{enumerate}

\item{GFT1\\}

Type: Functional, Dynamic, Automated
					
Initial State: Application is displaying the main menu page
					
Input: Cursor clicked on Start Game button
					
Output: New game is started and window is changed to reflect a new game state
					
How test will be performed: The robot class will place the cursor within the coordinates of the Start Game button and the robot will perform a left-click
					
\item{GTF2\\}

Type: Functional, Dynamic, Automated
					
Initial State: Within the game state
					
Input: Left/Right/Up/Down arrow key pressed
					
Output: Player moves in the respective direction (if path is clear)
					
How test will be performed: The robot class will virtually press the left/right/up/down arrow

\item{GTF3\\}

Type: Functional, Dynamic, Automated
					
Initial State: Player moving towards wall
					
Input: No input
					
Output: Player stops moving when coming in contact with the wall

How test will be performed: The robot class will output a line of text to the console indicating that the player has stopped due to collision with a wall

\item{GTF3\\}

Type: Functional, Dynamic, Automated
					
Initial State: 
					
Input: 
					
Output: 

How test will be performed: 

\item{GTF3\\}

Type: Functional, Dynamic, Automated
					
Initial State: 
					
Input: 
					
Output: 

How test will be performed: 

\item{GTF3\\}

Type: Functional, Dynamic, Automated
					
Initial State: 
					
Input: 
					
Output: 

How test will be performed: 
\item{GTF3\\}

Type: Functional, Dynamic, Automated
					
Initial State: 
					
Input: 
					
Output: 

How test will be performed: 

\item{GTF3\\}

Type: Functional, Dynamic, Automated
					
Initial State: 
					
Input: 
					
Output: 

How test will be performed: 

\item{GTF3\\}

Type: Functional, Dynamic, Automated
					
Initial State: 
					
Input: 
					
Output: 

How test will be performed: 

\item{GTF3\\}

Type: Functional, Dynamic, Automated
					
Initial State: 
					
Input: 
					
Output: 

How test will be performed: 

\item{GTF3\\}

Type: Functional, Dynamic, Automated
					
Initial State: 
					
Input: 
					
Output: 

How test will be performed: 

\item{GTF3\\}

Type: Functional, Dynamic, Automated
					
Initial State: 
					
Input: 
					
Output: 

How test will be performed: 

\item{GTF3\\}

Type: Functional, Dynamic, Automated
					
Initial State: 
					
Input: 
					
Output: 

How test will be performed: 

\item{GTF3\\}

Type: Functional, Dynamic, Automated
					
Initial State: 
					
Input: 
					
Output: 

How test will be performed: 

\item{GTF3\\}

Type: Functional, Dynamic, Automated
					
Initial State: 
					
Input: 
					
Output: 

How test will be performed: 


\end{enumerate}

\subsubsection{Area of Testing2}

...

\subsection{Tests for Non-functional Requirements}

\paragraph{}
This section of System Testing will consist of manual tests since it would be infeasible to implement automated tests for a majority of the subsections of Non-functional Requirements. Manual tests will be performed by users, therefore many of the tests will correspond with the Usability Survey so testers can ensure that these requirements are met.

\subsubsection{Look and Feel Requirements}
		
\paragraph{Application Layout}

\begin{enumerate}

\item{AL1\\}

Type: Functional, Dynamic, Manual
					
Initial State: Application is displaying the instructions page
					
Input: No user input
					
Output: Namcap instructions page
					
How test will be performed: User will look at the Namcap instructions page and ensure that the layout and color scheme of the display is similar (but not identical) to that of the original arcade Pacman
					
\item{AL2\\}

Type: Functional, Dynamic, Manual
					
Initial State: Application is displaying the game board with game entities
					
Input: No user input
					
Output: Namcap game board with game entities
					
How test will be performed: User will look at the Namcap game board with its game entities and ensure that the layout and color scheme of the display is similar (but not identical) to that of the original arcade Pacman

\end{enumerate}

\subsubsection{Usability and Humanity Requirements}

\paragraph{Understandable Objectives}

\begin{enumerate}

\item{UO1\\}

Type: Functional, Dynamic, Manual
					
Initial State: Application is displaying the instructions page
					
Input: No user input
					
Output: Namcap instructions page in English
					
How test will be performed: User will look at the Namcap instructions page and ensure that the instructions are in English
					
\item{UO2\\}

Type: Functional, Dynamic, Manual
					
Initial State: Application is displaying the instructions page
					
Input: No user input
					
Output: Namcap instructions page clearly defining objectives
					
How test will be performed: User will look at the Namcap instructions page and ensure that the objective of the game is clear before playing the game

\end{enumerate}

\paragraph{Understandable Gameplay}

\begin{enumerate}

\item{UG1\\}

Type: Functional, Dynamic, Manual
					
Initial State: Application is displaying the game board with game entities
					
Input: Custom user input for gameplay
					
Output: Game display updates accounting for movement, collision, scoring, and enemy mechanics
					
How test will be performed: User will play Namcap (one playthrough - 3 lives) and be able to successfully maneuver the Player through dots and enemies, reinforcing their understanding of the game's controls and objectives

\end{enumerate}

\subsubsection{Performance Requirements}

\paragraph{Response Time}

\begin{enumerate}

\item{RT1\\}

Type: Functional, Dynamic, Manual
					
Initial State: Application is displaying the game board with game entities
					
Input: Custom user input for gameplay
					
Output: Game display updates player movement based on keyboard input with a delay less than $\hyperref[tab:constants]{\Theta}$
					
How test will be performed: User will play Namcap and note down any instances where their player does not respond to their keyboard inputs within a delay of $\hyperref[tab:constants]{\Theta}$

\end{enumerate}

\paragraph{Unexpected Failures}

\begin{enumerate}

\item{UF1\\}

Type: Functional, Dynamic, Manual
					
Initial State: Application is displaying the game board with game entities
					
Input: Custom user input for gameplay testing
					
Output: Game display updates for user input and application does not crash
					
How test will be performed: User will play Namcap and note down any instances where during their testing, the application unexpectedly crashes (the number of crashes will be within $\hyperref[tab:constants]{\Omega}$ of total user tests)

\end{enumerate}

\subsubsection{Maintainability and Support Requirements}

\paragraph{Operating System Support\\}
The primary operating systems that Namcap is being tested on currently are Windows and Mac OS since these are the most readily available for testers.

\begin{enumerate}

\item{OSS1\\}

Type: Functional, Dynamic, Manual
					
Initial State: A Namcap jar file is available on a machine running Windows
					
Input: User runs the jar file
					
Output: Namcap starts up and game can be played
					
How test will be performed: User will run the jar file on their computer (any Windows version with a JVM) and should be able to successfully play the game (one playthrough)

\item{OSS2\\}

Type: Functional, Dynamic, Manual
					
Initial State: A Namcap jar file is available on a machine running Mac OS
					
Input: User runs the jar file
					
Output: Namcap starts up and game can be played
					
How test will be performed: User will run the jar file on their computer (any Mac OS version with a JVM) and should be able to successfully play the game (one playthrough)

\end{enumerate}

\subsubsection{Security Requirements}

...

\subsubsection{Cultural Requirements}

...

\subsubsection{Legal Requirements}

...

\subsubsection{Health and Safety Requirements}

...

\section{Tests for Proof of Concept}

\subsection{Area of Testing1}
		
\paragraph{Title for Test}

\begin{enumerate}

\item{test-id1\\}

Type: Functional, Dynamic, Manual, Static etc.
					
Initial State: 
					
Input: 
					
Output: 
					
How test will be performed: 
					
\item{test-id2\\}

Type: Functional, Dynamic, Manual, Static etc.
					
Initial State: 
					
Input: 
					
Output: 
					
How test will be performed: 

\end{enumerate}

\subsection{Area of Testing2}

...

	
\section{Comparison to Existing Implementation}	
				
\section{Unit Testing Plan}
		
\subsection{Unit testing of internal functions}
		
\subsection{Unit testing of output files}		

%\bibliographystyle{plainnat}

%\bibliography{TestPlan}

\newpage

\section{Appendix}

This is where you can place additional information.

\subsection{Symbolic Parameters}

The definition of the test cases will call for SYMBOLIC\_CONSTANTS; their values are defined in this section for easy maintenance.

\begin{table}[H]
\caption{\bf Symbolic Constants} \label{tab:constants}
\begin{tabularx}{\textwidth}{p{3cm}p{2cm}X}
\toprule {\bf Constant} & {\bf Value} & {\bf Description}\\
\midrule
$\Theta$ & 0.5 seconds & Response time between user actions and in-game operations\\
$\Omega$ & 1 \% & Percent of application tests that will cause unexpected failures in the application\\
\bottomrule
\end{tabularx}
\end{table}

\subsection{Usability Survey Questions?}

This is a section that would be appropriate for some teams.

\end{document}
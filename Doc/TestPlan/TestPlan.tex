\documentclass[12pt, titlepage]{article}

\usepackage{booktabs}
\usepackage{tabularx}
\usepackage{hyperref}
\hypersetup{
    colorlinks,
    citecolor=black,
    filecolor=black,
    linkcolor=red,
    urlcolor=blue
}
%\usepackage[round]{natbib}

\title{SE 3XA3: Test Plan\\Namcap}

\author{Team 2, VPB Game Studio
		\\ Prajvin Jalan (jalanp)
		\\ Vatsal Shukla (shuklv2)
		\\ Baltej Toor (toorbs)
}

\date{\today}

%% Comments

\usepackage{color}

\newif\ifcomments\commentstrue

\ifcomments
\newcommand{\authornote}[3]{\textcolor{#1}{[#3 ---#2]}}
\newcommand{\todo}[1]{\textcolor{red}{[TODO: #1]}}
\else
\newcommand{\authornote}[3]{}
\newcommand{\todo}[1]{}
\fi

\newcommand{\wss}[1]{\authornote{blue}{SS}{#1}}
\newcommand{\ds}[1]{\authornote{red}{DS}{#1}}
\newcommand{\mj}[1]{\authornote{red}{MSN}{#1}}
\newcommand{\cm}[1]{\authornote{red}{CM}{#1}}
\newcommand{\mh}[1]{\authornote{red}{MH}{#1}}

% team members should be added for each team, like the following
% all comments left by the TAs or the instructor should be addressed
% by a corresponding comment from the Team

\newcommand{\tm}[1]{\authornote{magenta}{Team}{#1}}


\begin{document}

\maketitle

\pagenumbering{roman}
\tableofcontents
\listoftables
\listoffigures

\begin{table}[bp]
\caption{\bf Revision History}
\begin{tabularx}{\textwidth}{p{3cm}p{2cm}X}
\toprule {\bf Date} & {\bf Version} & {\bf Notes}\\
\midrule
2016-10-30 & 1.0 & Addition of content to sections 1.1, 1.2 and 1.4\\
2016-10-30 & 1.1 & Addition of content to sections 2.1 and 2.2\\
Date 3 & 1.2 & Notes\\
\bottomrule
\end{tabularx}
\end{table}

\newpage

\pagenumbering{arabic}

This document is the test plan for the Pacman redevelopment project, Namcap. The test plan outlines the testin methodologies and techniques to be used when testing the functionalities and characteristics of the system and its component parts.

\section{General Information}

\subsection{Purpose}
\paragraph{}
The purpose of testing is to ensure that the developed implementation functions correctly and to address any areas where the system is vulnerable. Through the formal specification of the testing methods and verification techniques, testing the implementation becomes mroe reliable.

\subsection{Scope}
\paragraph{}
As Namcap is a redevelopment of a classic arcade game, the test plan will aim to formalize the various functionality testing techniques as well as the usability tests utilized in order to ensure that the implementation meets the given requirement. This document will explicitly detail the different methods and testing tools to be utilized for this project.

\subsection{Acronyms, Abbreviations, and Symbols}
	
\begin{table}[hbp]
\caption{\textbf{Table of Abbreviations}} \label{Table}

\begin{tabularx}{\textwidth}{p{3cm}X}
\toprule
\textbf{Abbreviation} & \textbf{Definition} \\
\midrule
Abbreviation1 & Definition1\\
Abbreviation2 & Definition2\\
\bottomrule
\end{tabularx}

\end{table}

\begin{table}[!htbp]
\caption{\textbf{Table of Definitions}} \label{Table}

\begin{tabularx}{\textwidth}{p{3cm}X}
\toprule
\textbf{Term} & \textbf{Definition}\\
\midrule
Term1 & Definition1\\
Term2 & Definition2\\
\bottomrule
\end{tabularx}

\end{table}	

\subsection{Overview of Document}
\paragraph{}
This document will describe the testing methodologies to be utilized to verify Namcap as an implementation and development. The test plan will outline all testing tools, schedules, automated and manual tests, tests to address the requirements for the application, unit tests, and any additional testing performed on the PoC and existing implementation.

\section{Plan}
	
\subsection{Software Description}
\paragraph{}
Namcap is a redevelopment of the classic 2D arcade game Pacman. The gameplay involves the player sprite moving through a 2D level attempting to acquire (collide with) as many dots as possible to increase the score. Enemy sprites (ghosts in the original) will move throughout the level and upon collision with the player will cause the player to lose a life and/or end the game. The player can however consume (collide with) a power up to send the enemies back to the center of the level (when collision occurs). The implementation covers these aspects of the core gameplay (scoring, collision, movement, and enemy mechanics).

\subsection{Test Team}
\paragraph{}
The test team for Namcap is comprised of Prajvin Jalan, Vatsal Shukla, and Baltej Toor.

\subsection{Automated Testing Approach}

\subsection{Testing Tools}

\subsection{Testing Schedule}
		
See Gantt Chart at the following url ...

\section{System Test Description}
	
\subsection{Tests for Functional Requirements}

\subsubsection{Area of Testing1}
		
\paragraph{Title for Test}

\begin{enumerate}

\item{test-id1\\}

Type: Functional, Dynamic, Manual, Static etc.
					
Initial State: 
					
Input: 
					
Output: 
					
How test will be performed: 
					
\item{test-id2\\}

Type: Functional, Dynamic, Manual, Static etc.
					
Initial State: 
					
Input: 
					
Output: 
					
How test will be performed: 

\end{enumerate}

\subsubsection{Area of Testing2}

...

\subsection{Tests for Nonfunctional Requirements}

\subsubsection{Area of Testing1}
		
\paragraph{Title for Test}

\begin{enumerate}

\item{test-id1\\}

Type: 
					
Initial State: 
					
Input/Condition: 
					
Output/Result: 
					
How test will be performed: 
					
\item{test-id2\\}

Type: Functional, Dynamic, Manual, Static etc.
					
Initial State: 
					
Input: 
					
Output: 
					
How test will be performed: 

\end{enumerate}

\subsubsection{Area of Testing2}

...

\section{Tests for Proof of Concept}

\subsection{Area of Testing1}
		
\paragraph{Title for Test}

\begin{enumerate}

\item{test-id1\\}

Type: Functional, Dynamic, Manual, Static etc.
					
Initial State: 
					
Input: 
					
Output: 
					
How test will be performed: 
					
\item{test-id2\\}

Type: Functional, Dynamic, Manual, Static etc.
					
Initial State: 
					
Input: 
					
Output: 
					
How test will be performed: 

\end{enumerate}

\subsection{Area of Testing2}

...

	
\section{Comparison to Existing Implementation}	
				
\section{Unit Testing Plan}
		
\subsection{Unit testing of internal functions}
		
\subsection{Unit testing of output files}		

%\bibliographystyle{plainnat}

%\bibliography{TestPlan}

\newpage

\section{Appendix}

This is where you can place additional information.

\subsection{Symbolic Parameters}

The definition of the test cases will call for SYMBOLIC\_CONSTANTS.
Their values are defined in this section for easy maintenance.

\subsection{Usability Survey Questions?}

This is a section that would be appropriate for some teams.

\end{document}
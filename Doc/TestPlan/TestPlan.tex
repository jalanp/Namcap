\documentclass[12pt, titlepage]{article}

\usepackage{booktabs}
\usepackage{tabularx}
\usepackage{hyperref}
\hypersetup{
    colorlinks,
    citecolor=black,
    filecolor=black,
    linkcolor=red,
    urlcolor=blue
}
%\usepackage[round]{natbib}
\usepackage{float}
\usepackage{mdframed}

\title{SE 3XA3: Test Plan\\Namcap}

\author{Team 2, VPB Game Studio
		\\ Prajvin Jalan (jalanp)
		\\ Vatsal Shukla (shuklv2)
		\\ Baltej Toor (toorbs)
}

\newcommand{\zeroToTen}{
\begin{tabularx}{4.40cm}{@{}p{0.40cm}p{0.40cm}p{0.40cm}p{0.40cm}p{0.40cm}p{0.40cm}p{0.40cm}p{0.40cm}p{0.40cm}p{0.40cm}p{0.40cm}@{}}
0 & 1 & 2 & 3 & 4 & 5 & 6 & 7 & 8 & 9 & 10
\end{tabularx}
}

\newcommand{\zeroToThree}{
\begin{tabularx}{1.20cm}{@{}p{0.30cm}p{0.30cm}p{0.30cm}p{0.30cm}@{}}
0 & 1 & 2 & 3+
\end{tabularx}
}

\newcommand{\yesNo}{
\begin{tabularx}{1.40cm}{@{}p{0.70cm}p{0.70cm}@{}}
Yes & No
\end{tabularx}
}

\date{\today}

%%% Comments

\usepackage{color}

\newif\ifcomments\commentstrue

\ifcomments
\newcommand{\authornote}[3]{\textcolor{#1}{[#3 ---#2]}}
\newcommand{\todo}[1]{\textcolor{red}{[TODO: #1]}}
\else
\newcommand{\authornote}[3]{}
\newcommand{\todo}[1]{}
\fi

\newcommand{\wss}[1]{\authornote{blue}{SS}{#1}}
\newcommand{\ds}[1]{\authornote{red}{DS}{#1}}
\newcommand{\mj}[1]{\authornote{red}{MSN}{#1}}
\newcommand{\cm}[1]{\authornote{red}{CM}{#1}}
\newcommand{\mh}[1]{\authornote{red}{MH}{#1}}

% team members should be added for each team, like the following
% all comments left by the TAs or the instructor should be addressed
% by a corresponding comment from the Team

\newcommand{\tm}[1]{\authornote{magenta}{Team}{#1}}


\begin{document}

\maketitle

\pagenumbering{roman}
\tableofcontents
\listoftables
\listoffigures

\begin{table}[h]
\caption{\bf Revision History}
\begin{tabularx}{\textwidth}{p{3cm}p{2cm}X}
\toprule {\bf Date} & {\bf Version} & {\bf Notes}\\
\midrule
2016-10-30 & 1.0 & Addition of content to sections 1.1, 1.2 and 1.4\\
2016-10-30 & 1.1 & Addition of content to sections 2.1 and 2.2\\
2016-10-30 & 1.2 & Addition of content to sections 2.3 and 2.4\\
2016-10-30 & 1.3 & Addition of content to section 3.2\\
2016-10-30 & 1.4 & Updates to content of subsections of 3.2 and 7.1\\
2016-10-30 & 1.5 & Completion of content for 3.2 and update to 7.1\\
2016-10-31 & 1.6 & Completion of content for 3.1\\
2016-10-31 & 1.6 & Completion of Section 7.2\\
2016-10-31 & 1.7 & Completion of content for 4\\
2016-10-31 & 1.8 & Completion of section 1.3; revision of sections 1, 2, and 4\\
2016-10-31 & 1.9 & Addition of section 6.1 and 6.2\\
2016-10-31 & 2.0 & Completion of section 5\\
2016-10-31 & 2.1 & Updates to Sections 3.1 and 6.1\\
\bottomrule
\end{tabularx}
\end{table}

\newpage

\pagenumbering{arabic}

This document is the test plan for the Pacman redevelopment project, Namcap. The test plan outlines the testing methodologies and techniques to be used when testing the functionalities and characteristics of the system and its component parts.

\section{General Information}

\subsection{Purpose}
\paragraph{}
The purpose of testing is to ensure that the developed implementation functions correctly and to address any areas where the system is vulnerable. Through the formal specification of the testing methods and verification techniques, testing the implementation becomes more reliable.

\subsection{Scope}
\paragraph{}
As Namcap is a redevelopment of a classic arcade game, the test plan will aim to formalize the various functionality testing techniques as well as the usability tests utilized in order to ensure that the implementation meets the given requirement. This document will explicitly detail the different methods and testing tools to be utilized for this project.

\subsection{Acronyms, Abbreviations, and Symbols}
	
\begin{table}[hbp]
\caption{\textbf{Table of Abbreviations}} \label{Table}

\begin{tabularx}{\textwidth}{p{3cm}X}
\toprule
\textbf{Abbreviation} & \textbf{Definition} \\
\midrule
PoC & Proof of Concept\\
GUI & Graphical User Interface\\
AI & Artificial Intelligence\\
JAR & Java Archive (file)\\
OS & Operating System\\
JVM & Java Virtual Machine\\
\bottomrule
\end{tabularx}

\end{table}

\begin{table}[!htbp]
\caption{\textbf{Table of Definitions}} \label{Table}

\begin{tabularx}{\textwidth}{p{3cm}X}
\toprule
\textbf{Term} & \textbf{Definition}\\
\midrule
Structural Testing & (White box testing) derived from the program's internal structure.\\
Functional Testing & (Black box testing) derived from a description of the program's function.\\
Dynamic Testing & Testing that requires program execution.\\
Static Testing & Testing that does not involve program execution.\\
Manual Testing & Testing that is conducted by people, by-hand testing.\\
Automated Testing & Testing that occurs automatically, usually set-up with testing frameworks.\\
\bottomrule
\end{tabularx}

\end{table}	

\subsection{Overview of Document}
\paragraph{}
This document will describe the testing methodologies to be utilized to verify Namcap as an implementation and development. The test plan will outline all testing tools, schedules, automated and manual tests, tests to address the requirements for the application, unit tests, and any additional testing performed on the PoC and existing implementation.

\section{Plan}
	
\subsection{Software Description}
\paragraph{}
Namcap is a redevelopment of the classic 2D arcade game Pacman. The gameplay involves the player sprite moving through a 2D level attempting to acquire (collide with) as many dots as possible to increase the score. Enemy sprites (ghosts in the original) will move throughout the level and upon collision with the player will cause the player to lose a life and/or end the game. The player can however consume (collide with) a power up to send the enemies back to the center of the level (when collision occurs). The implementation covers these aspects of the core gameplay (scoring, collision, movement, and enemy mechanics).

\subsection{Test Team}
\paragraph{}
The test team for Namcap is comprised of Prajvin Jalan, Vatsal Shukla, and Baltej Toor.

\subsection{Automated Testing Approach}
\paragraph{}
For the purposes of automated testing, the test team will use both JUnit and the built-in Robot class library. JUnit is a unit testing framework that will run automated tests for most logical components and any GUI components where applicable. The Robot class library will be used to automate testing that simulates user input. Based on the simulated user input, logical and GUI components will be tested to ensure that the appropriate collision and scoring responses occur.

\subsection{Testing Tools}
\paragraph{}
The automated unit testing tool JUnit and the Robot class library are the only testing tools that the team will use to verify the implementation.

\subsection{Testing Schedule}
		
Follow this link to the [\href{run:../DevelopmentPlan/NamcapGanttProject.gan}{Namcap Gantt Project}] (must have project file structure).

\section{System Test Description}
	
\subsection{Tests for Functional Requirements}

\subsubsection{Game Funcionality Testing}
		
\paragraph{}
A robot (automated) unit testing class will be implemented and used to test the mechanics of the game.

\begin{enumerate}

\item{GFT1\\}

Type: Functional, Dynamic, Automated
					
Initial State: Application is displaying the main menu page
					
Input: Cursor clicked on Start Game button
					
Output: New game is started and window is changed to reflect a new game state
					
How test will be performed: The robot class will place the cursor within the coordinates of the Start Game button and the robot will perform a left-click

\item{GFT11\\}

Type: Functional, Dynamic, Automated
					
Initial State: Within game state
					
Input: Cursor clicked on exit game
					
Output: Application must terminate

How test will be performed: The robot class will move the cursor to the exit game button and perform a left-click. If the applictaion closes, the test is successful

\end{enumerate}

\subparagraph{Player Movement/Collision Testing}

\begin{enumerate}

\item{GFT2\\}

Type: Functional, Dynamic, Automated
					
Initial State: Within the game state
					
Input: Arrow keys
					
Output: Player moves in the respective direction (if path is clear)
					
How test will be performed: The robot class will virtually press the left/right/up/down arrow

\item{GFT3\\}

Type: Functional, Dynamic, Automated
					
Initial State: Player comes in contact with wall
					
Input: No input
					
Output: Player stops moving when coming in contact with the wall

How test will be performed: The robot class will output a line of text to the console indicating that the player has stopped due to collision with a wall

\item{GFT4\\}

Type: Functional, Dynamic, Automated
					
Initial State: Player comes in contact with enemy
					
Input: No input
					
Output: If player has more than 1 life, decrement lives. If player has one life, end game.

How test will be performed: The robot class will navigate the payer towards an enemy until they collide.


\item{GFT6\\}

Type: Functional, Dynamic, Automated
					
Initial State: Player comes in contact with dots
					
Input: Arrow keys
					
Output: Dot disappears after collection

How test will be performed: The Robot class will navigate the player through the map and collect the dots. A function will be implemented to check if the dots disappear upon collection

\item{GFT7\\}

Type: Functional, Dynamic, Automated
					
Initial State: Player collects the big dot
					
Input: Arrow keys
					
Output: Big dot disappears after collection

How test will be performed: The robot class will navigate the player to the nearest big dot and collect it. A fucntion will be implemented to check if the big dot dissapears upon collection

\item{GFT8\\}

Type: Functional, Dynamic, Automated
					
Initial State: Player collects the big dot
					
Input: Arrow keys
					
Output: Player is able to collide with enemies

How test will be performed: The robot class will navigate the player to the nearest enemy, after colleting the big dot. A funciton will be implemented to check if the collision doesn't decrement the player's lives

\end{enumerate}

\subparagraph{Enemy Movement/Collision Testing}

\begin{enumerate}

\item{GFT5\\}

Type: Functional, Dynamic, Automated
					
Initial State: Within the game state
					
Input: No input
					
Output: Enemies move on a valid path

How test will be performed: A JUnit test will be implemented to fail whenever the enemy moves through an obstacle.

\item{GFT9\\}

Type: Functional, Dynamic, Automated
					
Initial State: Player collects the big dot
					
Input: No input
					
Output: Enemies change colour

How test will be performed: A JUnit test will be iplemented to check if the asset used to view the enemies has changed.

\item{GFT14\\}

Type: Functional, Dynamic, Automated
					
Initial State: Player collides with enemy after collection of big dot
					
Input: Arrow keys
					
Output: Enemy is removed from game and respawned back to their original cell

How test will be performed: A JUnit test will be implemented to check if the enemey's coordinates were reset to their original spot.

\end{enumerate}

\subparagraph{Scoring Testing}

\begin{enumerate}

\item{GFT10\\}

Type: Functional, Dynamic, Automated
					
Initial State: Player collects all dots
					
Input: Arrow keys
					
Output: Game over screen is activated

How test will be performed: A JUnit test will be implemented to check which JFrame is currently active

\item{GFT12\\}

Type: Functional, Dynamic, Automated
					
Initial State: Player collects dot
					
Input: Arrow keys
					
Output: The points are increased

How test will be performed: A JUnit test will be implemented to check if the player's points were increased

\item{GFT13\\}

Type: Functional, Dynamic, Automated
					
Initial State: Player collects big dot
					
Input: Arrow keys
					
Output: The points are increased at twice the rate

How test will be performed: A JUnit test will be implmented to check if the player's points were increased by twice the standard rate

\end{enumerate}

\subsection{Tests for Non-functional Requirements}

\paragraph{}
This section of System Testing will consist of manual tests since it would be infeasible to implement automated tests for a majority of the subsections of Non-functional Requirements. Manual tests will be performed by users, therefore many of the tests will correspond with the Usability Survey so testers can ensure that these requirements are met.

\subsubsection{Look and Feel Requirements}
		
\paragraph{Application Layout}

\begin{enumerate}

\item{AL1\\}

Type: Functional, Dynamic, Manual
					
Initial State: Application is displaying the instructions page
					
Input: No user input
					
Output: Namcap instructions page
					
How test will be performed: User will look at the Namcap instructions page and ensure that the layout and color scheme of the display is similar (but not identical) to that of the original arcade Pacman
					
\item{AL2\\}

Type: Functional, Dynamic, Manual
					
Initial State: Application is displaying the game board with game entities
					
Input: No user input
					
Output: Namcap game board with game entities
					
How test will be performed: User will look at the Namcap game board with its game entities and ensure that the layout and color scheme of the display is similar (but not identical) to that of the original arcade Pacman

\end{enumerate}

\subsubsection{Usability and Humanity Requirements}

\paragraph{Understandable Objectives}

\begin{enumerate}

\item{UO1\\}

Type: Functional, Dynamic, Manual
					
Initial State: Application is displaying the instructions page
					
Input: No user input
					
Output: Namcap instructions page in English
					
How test will be performed: User will look at the Namcap instructions page and ensure that the instructions are in English
					
\item{UO2\\}

Type: Functional, Dynamic, Manual
					
Initial State: Application is displaying the instructions page
					
Input: No user input
					
Output: Namcap instructions page clearly defining objectives
					
How test will be performed: User will look at the Namcap instructions page and ensure that the objective of the game is clear before playing the game

\end{enumerate}

\paragraph{Understandable Gameplay}

\begin{enumerate}

\item{UG1\\}

Type: Functional, Dynamic, Manual
					
Initial State: Application is displaying the game board with game entities
					
Input: Custom user input for gameplay
					
Output: Game display updates accounting for movement, collision, scoring, and enemy mechanics
					
How test will be performed: User will play Namcap (one playthrough - 3 lives) and be able to successfully maneuver the Player through dots and enemies, reinforcing their understanding of the game's controls and objectives

\end{enumerate}

\subsubsection{Performance Requirements}

\paragraph{Response Time}

\begin{enumerate}

\item{RT1\\}

Type: Functional, Dynamic, Manual
					
Initial State: Application is displaying the game board with game entities
					
Input: Custom user input for gameplay
					
Output: Game display updates player movement based on keyboard input with a delay less than $\hyperref[tab:constants]{\Theta}$
					
How test will be performed: User will play Namcap and note down any instances where their player does not respond to their keyboard inputs within a delay of $\hyperref[tab:constants]{\Theta}$

\end{enumerate}

\paragraph{Unexpected Failures}

\begin{enumerate}

\item{UF1\\}

Type: Functional, Dynamic, Manual
					
Initial State: Application is displaying the game board with game entities
					
Input: Custom user input for gameplay testing
					
Output: Game display updates for user input and application does not crash
					
How test will be performed: User will play Namcap and note down any instances where during their testing, the application unexpectedly crashes (the number of crashes will be within $\hyperref[tab:constants]{\Omega}$ of total user tests)

\end{enumerate}

\subsubsection{Maintainability and Support Requirements}

\paragraph{Operating System Support\\}
The primary operating systems that Namcap is being tested on currently are Windows and Mac OS since these are the most readily available for testers.

\begin{enumerate}

\item{OSS1\\}

Type: Functional, Dynamic, Manual
					
Initial State: A Namcap jar file is available on a machine running Windows
					
Input: User runs the jar file
					
Output: Namcap starts up and game can be played
					
How test will be performed: User will run the jar file on their computer (any Windows version with a JVM) and should be able to successfully play the game (one playthrough)

\item{OSS2\\}

Type: Functional, Dynamic, Manual
					
Initial State: A Namcap jar file is available on a machine running Mac OS
					
Input: User runs the jar file
					
Output: Namcap starts up and game can be played
					
How test will be performed: User will run the jar file on their computer (any Mac OS version with a JVM) and should be able to successfully play the game (one playthrough)

\end{enumerate}

\subsubsection{Security Requirements}

\paragraph{Open-Source Code}

\begin{enumerate}

\item{OSC1\\}

Type: Structural, Static, Manual
					
Initial State: Open-source repository is publicly accessible
					
Input: User accesses the open-source repository (visits the repository link)
					
Output: User is able to view the application source code
					
How test will be performed: User will visit the repository link and be able to view the application source code

\end{enumerate}

\subsubsection{Cultural Requirements}

\paragraph{Offensive Symbols or Text}

\begin{enumerate}

\item{OST1\\}

Type: Functional, Dynamic, Manual
					
Initial State: Application is displaying the instructions page
					
Input: No user input
					
Output: Namcap instructions page
					
How test will be performed: User will look at the Namcap instructions page and ensure that it contains no offensive symbols or text
					
\item{OST2\\}

Type: Functional, Dynamic, Manual
					
Initial State: Application is displaying the game board with game entities
					
Input: No user input
					
Output: Namcap game board with game entities
					
How test will be performed: User will look at the Namcap game board with its game entities and ensure that it contains no offensive symbols or text

\end{enumerate}

\subsubsection{Legal Requirements}

\paragraph{Legal Violation}

\begin{enumerate}

\item{LV1\\}

Type: Functional, Dynamic, Manual
					
Initial State: Application is displaying the instructions page
					
Input: No user input
					
Output: Namcap instructions page
					
How test will be performed: User will look at the Namcap instructions page and ensure that literary work and listed game entities have been altered such that the redevelopment is not too similar to the original Pacman
					
\item{LV2\\}

Type: Functional, Dynamic, Manual
					
Initial State: Application is displaying the game board with game entities
					
Input: No user input
					
Output: Namcap game board with game entities
					
How test will be performed: User will look at the Namcap game board with its game entities and ensure that literary work and game entities have been altered such that the redevelopment is not too similar to the original Pacman

\end{enumerate}

\subsubsection{Health and Safety Requirements}

\paragraph{Break Prompt}

\begin{enumerate}

\item{BP1\\}

Type: Functional, Dynamic, Manual
					
Initial State: Application is displaying the instructions page
					
Input: No user input, application runs for $\hyperref[tab:constants]{\Phi}$
					
Output: After $\hyperref[tab:constants]{\Phi}$, application pauses and prompts the user to take a break
					
How test will be performed: User will let the application run for $\hyperref[tab:constants]{\Phi}$ and ensure that when the time passes the application pauses appropriately and prompts them to take a break

\end{enumerate}

\section{Tests for Proof of Concept}

Before additional features can be added to the game, a basic proof of concept will be carried out to show that the project is feasible. 

\subsection{Significant Risk}

In order for the game to run, it must be succesfully compiled. Therefore, the game is intended to work on all operating systems as long as they have the latest version on Java installed. However, this can be a risk if the game is unable to run on some operating systems.

\subsection{Demonstration Plan}

For the proof of concept, a working prototype of the game will be produced that will run by opening a JAR file on any operating system with Java installed. The prototype will consist of the game which implements a player and enemy movement, collision detection, points system, and boundary detection.

	The game will consist of a basic Pacman map layout in which a player character and enemy character will exist. The player and enemy character will start at a specific point on the map. Neither the player nor the enemy can pass through the walls on the map. The player will be able to collect the dots to increment the game points. Upon collision with the enemy, the game will end and output the final score.

	The player character will be represented by a pacman .png file that will be located in the assets package within the project. The enemy character will be represented by a ghost .png file that will also be located in the assets package. The player will be controlled by the user by using the arrow keys on the keyboard.

	The enemy will have a simple AI programmed for the proof of concept. The AI will allow the enemy to move in the direction they are currently moving unless it hits a barrier. Then, the AI will randomly generate a direction for the enemy to move in. The enemy will move in the new generated direction if the path is clear, else, another direction is generated.
		
\paragraph{Proof of Concept Testing}

Refer to the Tests for Functional Requirements section for the following PoC tests:

\begin{enumerate}

\item{PoCT1\\}

Type: Functional, Dynamic, Manual
					
Refer to: GFT1
					
Description: The app can be opened


\item{PoCT2\\}

Type: Functional, Dynamic, Manual
					
Refer to: GFT2
					
Description: The player can be moved


\item{PoCT3\\}

Type: Functional, Dynamic, Manual
					
Refer to: GFT3
					
Description: Player cannot move through walls


\item{PoCT4\\}

Type: Functional, Dynamic, Manual
					
Refer to: GFT4
					
Description: The game will exit when player collides with enemy


\item{PoCT5\\}

Type: Functional, Dynamic, Manual
					
Refer to: GFT5
					
Description: Enemies move on a valid path


\item{PoCT6\\}

Type: Functional, Dynamic, Manual
					
Refer to: GFT6, GFT12
					
Description: The player can collect dots and increase points


\item{PoCT7\\}

Type: Functional, Dynamic, Manual
					
Refer to: GFT10
					
Description: Game over when all dots are collected

					
\end{enumerate}
	
\section{Comparison to Existing Implementation}	

To compare our implementatin with the exisiting implementation, we will manually perform our system tests on the existing project and compare the outputs. If the outputs are similar/same, we have implemented the project correctly. 

Refer to the System Test Description section for the following system test IDs.
These tests will be performed on the existing implementation in parallel with the developed implementation:

\begin{itemize}

\item{}
GFT2
\item{}
GFT3
\item{}
GFT4
\item{}
GFT6
\item{}
GFT7
\item{}
GFT8
\item{}
GFT9
\item{}
GFT10
\item{}
GFT12
\item{}
GFT13
\item{}
GFT14
\item{}
RT1
\item{}
UF1

\end{itemize}
				
\section{Unit Testing Plan}
		
\subsection{Unit testing of internal functions}

\paragraph{}
A JUnit testing class will be implemented and used to test specific aspects of the application. These tests will be more specific than the functional tests and will address methods within the source code.

\begin{enumerate}

\item{UT1\\}

Type: Unit, Dynamic, Automated
					
Initial State: Application is displaying the main menu page
					
Input: Start Game button action is performed
					
Output: New game is started and window is changed to reflect a new game state (Pass or fail for JUnit method)
					
How test will be performed: A JUnit method will test the method that opens a new frame containing the game and confirm that the new window has the appropriate game interface

\item{UT2\\}

Type: Unit, Dynamic, Automated
					
Initial State: Player is in starting position at the start of the game
					
Input: Current X position of the player
					
Output: Pass or fail for JUnit method
					
How test will be performed: A JUnit method will test the player's getCurrentX method to ensure their current X position equals their initial start position of the game

\item{UT3\\}

Type: Unit, Dynamic, Automated
					
Initial State: Player is in starting position at the start of the game
					
Input: Current Y position of the player
					
Output: Pass or fail for JUnit method
					
How test will be performed: A JUnit method will test the player's getCurrentY method to ensure their current Y position equals their initial start position of the game

\item{UT4\\}

Type: Unit, Dynamic, Automated
					
Initial State: Player is in starting position at the start of the game
					
Input: Current direction of the player
					
Output: Pass or fail for JUnit method
					
How test will be performed: A JUnit method will test the player's getCurrentDirection method to ensure their current direction equals the direction of their initial start position of the game

\item{UT5\\}

Type: Unit, Dynamic, Automated
					
Initial State: Player is in starting position at the start of the game
					
Input: PlayerX, PlayerY, EnemyX, EnemyY
					
Output: End of game (Pass or fail for JUnit method)
					
How test will be performed: A JUnit method will set the player's position to be within collision range of an enemy, and the checkCollision method will be tested to ensure it properly checks player-to-enemy collision and ends the game

\item{UT6\\}

Type: Unit, Dynamic, Automated
					
Initial State: Player is in starting position at the start of the game, all dots are on map
					
Input: PlayerX, PlayerY (different parts of the map grid)
					
Output: Increase in score and dot disappearance (Pass or fail for JUnit method)
					
How test will be performed: A JUnit method will test the player's checkDot method to ensure that when they are on a dot that their score is increased and the dot has disappeared (board's getDot method)

\item{UT7\\}

Type: Unit, Dynamic, Automated
					
Initial State: Player is in starting position at the start of the game, all dots on map are clear except one
					
Input: PlayerX, PlayerY
					
Output: End of game (Pass or fail for JUnit method)
					
How test will be performed: A JUnit method will test the player's checkDot method to ensure that when they collect the final dot that the endGame method is called and the player wins the game

\item{UT8\\}

Type: Unit, Dynamic, Automated
					
Initial State: Player is in starting position at the start of the game
					
Input: X and Y positions for barrier and non-barrier positions
					
Output: True or false for barriers (Pass or fail for JUnit method)
					
How test will be performed: A JUnit method will test the player's checkBarrier method to ensure that when a position is checked, the appropriate response is received for if barriers exist or not

\item{UT9\\}

Type: Unit, Dynamic, Automated
					
Initial State: Board is created with only barrier and dot entities
					
Input: X and Y positions for barrier locations
					
Output: Expected true for all barrier locations (Pass or fail for JUnit method)
					
How test will be performed: A JUnit method will test the board's getBarrier method to ensure that barriers exist in all the appropriate locations

\item{UT10\\}

Type: Unit, Dynamic, Automated
					
Initial State: Board is created with only barrier and dot entities
					
Input: X and Y positions for locations without barriers
					
Output: True or false for new barrier locations (Pass or fail for JUnit method)
					
How test will be performed: A JUnit method will test the board's updateBarrier method to ensure that barriers can be successfully added to specific locations on the map (getBarrier to check)

\item{UT11\\}

Type: Unit, Dynamic, Automated
					
Initial State: Board is created with only barrier and dot entities
					
Input: X and Y positions for dot locations
					
Output: Expected true for all dot locations (Pass or fail for JUnit method)
					
How test will be performed: A JUnit method will test the board's getDot method to ensure that dots exist in all the appropriate locations

\item{UT12\\}

Type: Unit, Dynamic, Automated
					
Initial State: Board is created with only barrier and dot entities
					
Input: X and Y positions for locations without dots
					
Output: Integer value for dot locations (type of dot that exists here) (Pass or fail for JUnit method)
					
How test will be performed: A JUnit method will test the board's updateDot method to ensure that dots can be successfully added to specific locations on the map (getDot to check)

\item{UT13\\}

Type: Unit, Dynamic, Automated
					
Initial State: Board is created with only barrier and dot entities
					
Input: X and Y positions for locations without dots
					
Output: Integer value for dot locations (type of dot that exists here) (Pass or fail for JUnit method)
					
How test will be performed: A JUnit method will test the board's updateDot method to ensure that dots can be successfully added to specific locations on the map (getDot to check)

\end{enumerate}
		
\subsection{Unit testing of output files}	

\begin{enumerate}

\item{UTF1\\}

Type: Unit, Dynamic, Automated
					
Initial State: Application is in gameplay state
					
Input: High score within game and High Score stored in text file
					
Output: Pass or fail for JUnit method
					
How test will be performed: A JUnit class will test the method that reads a high score from the text file and check whether or not the in game high score matches that of the text file

\end{enumerate}	

%\bibliographystyle{plainnat}

%\bibliography{TestPlan}

\newpage

\section{Appendix}

\subsection{Symbolic Parameters}

%The definition of the test cases will call for SYMBOLIC\_CONSTANTS; their values are defined in this section for easy maintenance.

\begin{table}[H]
\caption{\bf Symbolic Constants} \label{tab:constants}
\begin{tabularx}{\textwidth}{p{3cm}p{2cm}X}
\toprule {\bf Constant} & {\bf Value} & {\bf Description}\\
\midrule
$\Theta$ & 0.5 seconds & Response time between user actions and in-game operations\\
$\Omega$ & 1 \% & Percent of application tests that will cause unexpected failures in the application\\
$\Phi$ & 2 hours & Duration that application runs for before prompting user to take a break\\
\bottomrule
\end{tabularx}
\end{table}


\subsection{Usability Survey Questions}

The following \hyperref[fig:survey]{usability survey} will be filled out by users to evaluate playability of the game and test for some non-functional requirements where manual testing is required.

\newpage
\begin{figure}
\label{fig:survey}
\caption{User Survey}
\begin{mdframed}[linewidth=1pt]

\begin{center}
{\bf \large Usability Survey Questions}\\[\baselineskip]
\end{center}

\noindent Please complete the following survey as you play through the game. The first section covers usability of the application and your thoughts on the game in terms of general entertainment. The second section helps us test our game to ensure we can provide a stable application for the public.

\begin{center}
\noindent {\bf Playability}\\[\baselineskip]
\end{center}

\noindent \begin{tabularx}{\textwidth}{@{}p{3.4cm}X@{}}
{\bf Entertainment:} & \zeroToTen \\
& {\small [ where 10 is most entertaining ]}\\[\baselineskip]
{\bf Interface:} & \zeroToTen \\
& {\small [ where 10 is most user-friendly ]}\\[\baselineskip]
{\bf Game Difficulty:} & \zeroToTen \\
& {\small [ where 10 is most difficult ]}\\[\baselineskip]
\end{tabularx}

\begin{center}
\noindent {\bf User Testing}\\[\baselineskip]
\end{center}

\noindent \begin{tabularx}{\textwidth}{p{8.5cm}p{0.2cm}X}
{\small Does the instructions page look identical to that of the original Pacman?} & & \yesNo \\
{\small Does the game interface look identical to that of the original Pacman?} & & \yesNo \\
{\small Does the instructions page clearly define the objective of Namcap?} & & \yesNo \\
{\small Are you able to fully grasp the overall gameplay after one playthrough (3 lives)?} & & \yesNo \\
{\small Do you notice any offensive symbols or text in any part of the game?} & & \yesNo \\
{\small How many times did the game unexpectedly crash?} & & \zeroToThree \\
{\small Were there instances where your key presses had a response delay of more than 1 second?} & & \zeroToThree \\
\end{tabularx}

\end{mdframed}

\end{figure}

\end{document}